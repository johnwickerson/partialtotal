\documentclass[10pt,a4paper]{article}
\usepackage{a4wide}

\usepackage{ifthen}

\usepackage[T1]{fontenc}
\usepackage[utf8x]{inputenc}
\DeclareUnicodeCharacter{8656}{\ensuremath{\Leftarrow}}
\DeclareUnicodeCharacter{8614}{\ensuremath{{\mapsto}}}
\DeclareUnicodeCharacter{8871}{\ensuremath{\models}}
\DeclareUnicodeCharacter{963}{\ensuremath{\sigma}}
\DeclareUnicodeCharacter{956}{\ensuremath{\mu}}
\DeclareUnicodeCharacter{957}{\ensuremath{\nu}}


\usepackage{amsmath}
\usepackage{amssymb}
\usepackage{amsthm}
\usepackage{mathpartir}
\usepackage{johnproof}
\usepackage{verbatim}

% THEOREMS
\theoremstyle{theorem}
\newtheorem{todo}{Todo}
\theoremstyle{definition}
\newtheorem{theorem}{Theorem}
\newtheorem*{conjecture}{Conjecture}
\newtheorem{lem}[theorem]{Lemma}
\newtheorem{cor}[theorem]{Corollary}
\newtheorem{prop}[theorem]{Proposition}
\newtheorem{definition}[theorem]{Definition}
\newtheorem{rem}[theorem]{Remark}
\newtheorem{question}[theorem]{Question}
\newtheorem{problem}[theorem]{Problem}
\newtheorem{example}[theorem]{Example}
\newtheorem{exercise}[theorem]{Exercise}

\title{\vspace{-60pt} Partial and total correctness as greatest and least fixpoints}
\author{John Wickerson}
\date{25th November 2009}

\newcommand{\eqdef}{=_{\mathrm{def}}}
\newcommand{\prov}[1][]{\ifthenelse{\equal{#1}{}}{\vdash}{\vdash_{\mbox{\sf\scriptsize #1}}}}
\newcommand{\seqspec}[3]{\{#1\}\,#2\,\{#3\}}
\newcommand{\totalspec}[3]{[#1]\,#2\,[#3]}

\renewcommand{\parallel}{\mathbin{\mathsf{ll}}}
\newcommand{\semicolon}{\mathbin{\mathbf{;}}}
\newcommand{\Skip}{\mathop{\mathtt{skip}}}

\begin{document}
\maketitle
\thispagestyle{empty}

\begin{abstract}
This short note explains how partial and total correctness triples can be characterised as the greatest and least fixpoints, respectively, of the same function.
\end{abstract}

Suppose we have a small-step transition relation $→$ between configurations. Configurations comprise a command $C$ and a state σ. Given an initial configuration $(C_0,σ_0)$, we can produce a transition system corresponding to all the possible computations that can arise. A simple one is as follows:
\[
\begin{array}{c}
(\texttt{x++} \parallel \texttt{y++},\{{\tt x}↦0,{\tt y}↦2\}) \\
\swarrow \hspace{40pt} \searrow \\
({\tt skip} \parallel \texttt{y++},\{{\tt x}↦1,{\tt y}↦2\}) \qquad \qquad (\texttt{x++} \parallel {\tt skip},\{{\tt x}↦0,{\tt y}↦3\}) \\
\searrow \hspace{40pt} \swarrow \\
({\tt skip}\parallel {\tt skip},\{{\tt x}↦1,{\tt y}↦3\}) \\
↓\\
({\tt skip},\{{\tt x}↦1,{\tt y}↦3\})
\end{array}
\]

We can use modal-μ calculus to describe such transition systems. We shall consider modal-μ assertions with the following, slightly reduced, syntax:
\[
p ::= S \mid \top \mid \bot \mid p ⇒ p \mid p ∧ p \mid \square p \mid \diamond p \mid μX\ldotp p \mid νX\ldotp p \mid X
\]
where $S$ is a set of configurations and $μX\ldotp p$ and $νX\ldotp p$ are the least and greatest assertions such that $X=p$, where $p$ contains $X$. The satisfaction relation $(C,σ) ⊧ p$ is defined as follows:
\[
\begin{array}{lcl}
(C,σ)⊧S & ⇔ & (C,σ)\in S \\
(C,σ)⊧\top & ⇔ & \mathit{true} \\
(C,σ)⊧\bot & ⇔ & \mathit{false} \\
(C,σ)⊧p_1⇒p_2 & ⇔ & (C,σ)⊧p_1 ⇒ (C,σ)⊧p_2 \\
(C,σ)⊧p_1∧p_2 & ⇔ & (C,σ)⊧p_1 ∧ (C,σ)⊧p_2 \\
(C,σ)⊧\square p & ⇔ & ∀C',σ'\ldotp (C,σ)→(C',σ') ⇒ (C',σ')⊧p \\
(C,σ)⊧\diamond p & ⇔ & ∃C',σ'\ldotp (C,σ)→(C',σ') ∧ (C',σ')⊧p \\
(C,σ)⊧μX\ldotp p & ⇔ & (C,σ)\in \bigcap\{S\mid p[S/X] ⊆ S\} \\
(C,σ)⊧νX\ldotp p & ⇔ & (C,σ)\in \bigcup\{S\mid S ⊆ p[S/X]\}
\end{array}
\]
Let $P$ and $Q$ range over sets of states. Understand $(C,P)$ as a shorthand for $\{(C,σ)\mid σ\in P\}$, and let $(C,P)⊧p$ mean $∀σ\in P\ldotp (C,σ)⊧p$.

\begin{definition}[Termination] A configuration $(C,σ)$ is \emph{stuck} if no further reductions are possible. Moreover, the configuration is \emph{terminal} when $C={\tt skip}$.
\end{definition}

\begin{definition}[Partial correctness] $\seqspec{P}C{Q}$ means that whenever $C$ is executed from an initial state in $P$, then in every terminal configuration it reaches, the state is in $Q$.
\end{definition}

\begin{definition}[Total correctness] $\totalspec{P}C{Q}$ means that whenever $C$ is executed from an initial state in $P$, then it reaches a terminal configuration, and in every terminal configuration it reaches, the state is in $Q$.
\end{definition}

\begin{conjecture}Partial correctness and total correctness are characterised by the following modal-μ assertions.
\[
\begin{array}{rcl}
\seqspec{P}C{Q} & ⇔ & (C,P) ⊧ νX\ldotp \square X ∧ (\square\bot ⇒ ({\tt skip},Q)) \\
\totalspec{P}C{Q} & ⇔ & (C,P) ⊧ μX\ldotp \square X ∧ (\square\bot ⇒ ({\tt skip},Q)) 
\end{array}
\]
\end{conjecture}
\begin{proof}To do.\end{proof}

\end{document}